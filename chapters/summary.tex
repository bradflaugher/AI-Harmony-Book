\chapter*{Chapter Summaries}
\addcontentsline{toc}{chapter}{Chapter Summaries} % Add the preface to the table of contents as a chapter

\section*{1 History: The End of Good Old-Fashioned Artificial Intelligence}

Avoid the term "AI" and instead differentiate between rules-based programming and deep learning to prevent confusion. Good Old-Fashioned AI (rules-based AI) is challenging to create and maintain, while modern deep learning techniques use data to train models and can be as good as the data they are trained with. Good Old-Fashioned AI is not well-suited for many complex tasks, like translation and object detection.

\section*{2 The Regression Theory of Everything}

Deep learning models are large unscientific regression systems that map input data to output data. They are complex, deterministic and exhibit chaotic behavior, making their inner workings functionally unknowable and difficult to test. Multicollinearity and feature importance can only be understood with a high level of statistical error. Getting good data to train with is crucial for machine learning engineers to train good models. While deep learning models can have impressive and useful outputs, it's important to acknowledge their failures and limitations, which can be encouraged by users, managers and investors.

\section*{3 Creativity and Decision Making with Deep Learning Models}

Users should consider the quality and appropriateness of the data the model is trained on and ask questions about how and when the data was collected and cleaned up. Deep learning models are not sentient creative creatures and are deterministic systems that can only generate outputs based on their training data. Users should also avoid allowing models do everything and manage the models cautiously and carefully.

\section*{4 Model Cards and Case Studies}

An analysis of 30 machine learning models, from online dating, stock trading, threat detection and more.

\section*{5 Self-Driving with Statistics}

A discussion of the limitations of deep learning models and how they can be used and controlled in self-driving cars. A roadmap of likely developments and changes in infrastructure and control systems to support fully autonomous vehicles.

\section*{6 Unplugging Skynet}

An analysis of autonomous weapons, their risks and development. 

\section*{7 Revolutionary for Whom?}

An analysis of the past and future of work and automation, and a framework for understanding how daily tasks and careers will evolve over time.
