\setchapterpreamble[u]{\margintoc}
\chapter{Revolutionary for Whom?}
\labch{rev}

\textit{"The inhabitant of London could order by telephone, sipping his morning tea in bed, the various products of the whole earth -- he could at the same time and by the same means adventure his wealth in the natural resources and new enterprise of any quarter of the world -- he could secure forthwith, if he wished, cheap and comfortable means of transit to any country or climate without passport or other formality."} John Maynard Keynes, 1920 \cite{Keynes2012}

\section{Assistance from Assistants}

\textit{"Living off the wits of his subordinates - maybe that's leadership these days"}\cite{Lecarre}

You know, the whole concept of having an assistant has changed quite a bit over the years. In the old days, a well-to-do family might have employed a butler to help manage their estate, while nowadays, folks can simply use digital assistants like Siri or ChatGPT. But that's not all! There's also an interesting hybrid of sorts that has emerged in recent times – the Indian Virtual Assistant.

Now, if you think about it, traditional butlers were a sort of luxury. They cost a pretty penny and were thus reserved for the upper crust of society. On the other hand, these digital assistants – Siri, ChatGPT, and the like – well, they're pretty affordable, and just about anyone with a smartphone can access them. In a way, they've democratized assistance.

But there's another option that lies somewhere in between: the Indian Virtual Assistant. These human assistants, often based in India, provide remote support at a fraction of the cost of an in-person butler. They offer the human touch that digital assistants can't quite replicate, while still maintaining a certain level of affordability.

Now, don't get me wrong – I'm not saying digital assistants are bad. They're incredibly useful! But they've got some drawbacks, too. One concern is the potential for hacking and bugs. These AI-driven helpers are connected to the internet, and that means some folks with malicious intentions might try to break in and steal sensitive data. Plus, as impressive as machine learning can be, it's not perfect. Sometimes, these digital assistants might give you an answer that's a bit... off.

So, who might steer clear of AI assistants, despite their affordability and widespread availability? Well, for one, there are people who value their privacy and would rather not have a potential security risk handling their affairs. Then there are those who simply find comfort in human interaction, even if it means shelling out a little extra for a human assistant.

At the end of the day, the world of assistance has seen quite a revolution. As we navigate this new landscape, it's important to think about the trade-offs between cost, convenience, security, and the value of human expertise in the age of artificial intelligence.

\section{The Limitations of AI-generated Content}

\textit{"Another definition of modernity: conversations can be more and more completely reconstructed with clips from other conversations taking place at the same time on the planet.", "You are alive in inverse proportion to the density of cliches in your writing."}\cite{procrustes}

When it comes to AI-generated content, there's a certain fascination with the way machines can replicate human-like conversations. But are these AI-generated conversations truly akin to the ones between humans? Well, not exactly.

Let's take ChatGPT as an example. A recent article described it as a "Blurry JPEG of the Web" \cite{newyorkerChatGPTBlurry}, and that's quite an apt description. While AI models like ChatGPT are capable of generating coherent and contextually relevant responses, they often lack the depth and originality that genuine human conversations possess. In essence, AI-generated content can sometimes feel like a collage of borrowed ideas and phrases, strung together to mimic a conversation, but lacking the unique perspective and spontaneity that real human interaction entails.

Now, if we compare AI-generated content to traditional search engines, we can see some interesting differences. A search engine retrieves information from the vast expanse of the internet, presenting it to the user for their interpretation. An AI like ChatGPT, on the other hand, processes and generates content using a sophisticated regression model. But here's the catch: it might not always be right.

\textit{"(Traditional) search engines are databases, organized collections of data that can be stored, updated, and retrieved at will. (Traditional) search engines are indexes. a form of database, that connect things like keywords to URLs; they can be swiftly updated, incrementally, bit by bit (as when you update a phone number in the database that holds your contacts).}

\textit{Large language models do something very different: they are not databases; they are text predictors, turbocharged versions of autocomplete. Fundamentally, what they learn are relationships between bits of text, like words, phrases, even whole sentences. And they use those relationships to predict other bits of text. And then they do something almost magical: they paraphrase those bits of texts, almost like a thesaurus but much much better. But as they do so, as they glom stuff together, something often gets lost in translation: which bits of text do and do not truly belong together."} Gary Marcus, 2023 \cite{marcus2023}

In this sense, AI-generated content can be thought of as a creative interpretation of the information available to it. While it can provide valuable insights and answers, its limitations stem from its inability to truly comprehend the nuances of human conversation and thought. Consequently, the content produced may occasionally fall short in terms of accuracy, authenticity, or originality.

Therefore, as we continue to integrate AI-generated content into our lives, it's crucial to remember that these AI models, while impressive, are not perfect. They can offer valuable assistance, but they should not be treated as infallible sources of information or as substitutes for genuine human interaction.

\section{The AI Advantage in the Workplace}

There's a long-standing belief that human intelligence is unrivaled, and while it's true that humans possess unique capabilities, it's important to recognize the immense value AI can bring to the workplace.

Consider the humble calculator, for instance. Before its invention, humans would manually perform complex calculations, a process that was not only time-consuming but also prone to errors. The introduction of calculators transformed the way we approached mathematics, allowing us to quickly and accurately solve problems. Similarly, AI has the potential to revolutionize the way we work by handling tasks that would otherwise require significant time and effort from humans.

By embracing AI in daily tasks and careers, we can delegate responsibilities that computers excel at, such as processing large datasets, pattern recognition, and even language translation. This not only improves efficiency and accuracy but also frees up human workers to focus on tasks that require creativity, empathy, and critical thinking – areas where AI still falls short.

The key to reaping the benefits of AI in the workplace lies in finding the right balance between human expertise and AI capabilities. Rather than viewing AI as a threat to human intelligence, we should approach it as a powerful tool that complements and enhances our own skills. By working in tandem with AI, we can unlock new levels of productivity, innovation, and growth.

Ultimately, the integration of AI into the workplace offers countless opportunities for both individuals and organizations to thrive. By challenging the notion of superior human intelligence and acknowledging the strengths of AI, we can forge a harmonious partnership that drives success in the ever-evolving world of business.

section{The "Organic Content" Market}

As AI becomes increasingly prevalent in content production, we may witness the emergence of a new market segment: handcrafted, 100% organic content that is AI-free. This trend would be akin to the demand for organic, artisanal products in other industries, where consumers seek authenticity and the human touch.

The appeal of organic content lies in its perceived originality, creativity, and the assurance that it is the result of genuine human effort. This type of content may be highly sought after in industries such as journalism, literature, and the arts, where the value of human expression and unique perspectives is paramount.

However, the rise of this market also brings with it the potential for fraud. As AI-generated content becomes increasingly sophisticated and indistinguishable from human-generated content, distinguishing between the two may become a significant challenge. Unscrupulous individuals may attempt to pass off AI-generated content as organic, capitalizing on the demand for authentic human creations.

In order to combat fraud in the organic content market, it will be crucial to develop reliable methods for verifying the authenticity of content. This could include the use of digital signatures, blockchain technology, or other innovative solutions that provide a clear and tamper-proof record of a content's origin.

Moreover, fostering a culture of transparency and accountability within the content creation industry will be essential. By setting high ethical standards and encouraging open communication about the use of AI in content production, both creators and consumers can work together to ensure that the value of genuine human expression is preserved and celebrated.

\section{Worse, but still on top}

Krohn Monkeys and AI Superiority \cite{KrohnTED}

\section{Dead Inside}

\textit{"If you know, in the morning, what your day looks like with any precision you are a little bit dead - the more precision the more dead you are."}\cite{procrustes}

\section{Take it, or Be Left Behind}

point by point refutaiton of this idiotic letter. \cite{dumbestletter}

Should we let machines flood our information channels with propaganda and untruth? Just stop getting your news from social media, subscribe to the New York Times or WSJ and call it a day, this is already a problem with state actors flooding the zone with propaganda.

Should we automate away all the jobs, including the fulfilling ones? If you find automatable work fulfilling, start an Etsy shop and tell everyone you did everything by hand.... some people even sharpen pencils by hand, but stop pretending you can regulate this away. 

Should we develop nonhuman minds that might eventually outnumber, outsmart, obsolete and replace us? This is silly, Can they really outsmart us? Can they really replace us? Computers have been better at many tasks than us for forever, just because they can score better than 90 percent of humanity on the SAT does not mean that they can replace us. 

Should we risk loss of control of our civilization? WTF does this mean? Who controls it now?

